% Options for packages loaded elsewhere
% Options for packages loaded elsewhere
\PassOptionsToPackage{unicode}{hyperref}
\PassOptionsToPackage{hyphens}{url}
\PassOptionsToPackage{dvipsnames,svgnames,x11names}{xcolor}
%
\documentclass[
  letterpaper,
  DIV=11,
  numbers=noendperiod]{scrartcl}
\usepackage{xcolor}
\usepackage[margin=1in]{geometry}
\usepackage{amsmath,amssymb}
\setcounter{secnumdepth}{2}
\usepackage{iftex}
\ifPDFTeX
  \usepackage[T1]{fontenc}
  \usepackage[utf8]{inputenc}
  \usepackage{textcomp} % provide euro and other symbols
\else % if luatex or xetex
  \usepackage{unicode-math} % this also loads fontspec
  \defaultfontfeatures{Scale=MatchLowercase}
  \defaultfontfeatures[\rmfamily]{Ligatures=TeX,Scale=1}
\fi
\usepackage{lmodern}
\ifPDFTeX\else
  % xetex/luatex font selection
\fi
% Use upquote if available, for straight quotes in verbatim environments
\IfFileExists{upquote.sty}{\usepackage{upquote}}{}
\IfFileExists{microtype.sty}{% use microtype if available
  \usepackage[]{microtype}
  \UseMicrotypeSet[protrusion]{basicmath} % disable protrusion for tt fonts
}{}
\makeatletter
\@ifundefined{KOMAClassName}{% if non-KOMA class
  \IfFileExists{parskip.sty}{%
    \usepackage{parskip}
  }{% else
    \setlength{\parindent}{0pt}
    \setlength{\parskip}{6pt plus 2pt minus 1pt}}
}{% if KOMA class
  \KOMAoptions{parskip=half}}
\makeatother
% Make \paragraph and \subparagraph free-standing
\makeatletter
\ifx\paragraph\undefined\else
  \let\oldparagraph\paragraph
  \renewcommand{\paragraph}{
    \@ifstar
      \xxxParagraphStar
      \xxxParagraphNoStar
  }
  \newcommand{\xxxParagraphStar}[1]{\oldparagraph*{#1}\mbox{}}
  \newcommand{\xxxParagraphNoStar}[1]{\oldparagraph{#1}\mbox{}}
\fi
\ifx\subparagraph\undefined\else
  \let\oldsubparagraph\subparagraph
  \renewcommand{\subparagraph}{
    \@ifstar
      \xxxSubParagraphStar
      \xxxSubParagraphNoStar
  }
  \newcommand{\xxxSubParagraphStar}[1]{\oldsubparagraph*{#1}\mbox{}}
  \newcommand{\xxxSubParagraphNoStar}[1]{\oldsubparagraph{#1}\mbox{}}
\fi
\makeatother


\usepackage{longtable,booktabs,array}
\usepackage{calc} % for calculating minipage widths
% Correct order of tables after \paragraph or \subparagraph
\usepackage{etoolbox}
\makeatletter
\patchcmd\longtable{\par}{\if@noskipsec\mbox{}\fi\par}{}{}
\makeatother
% Allow footnotes in longtable head/foot
\IfFileExists{footnotehyper.sty}{\usepackage{footnotehyper}}{\usepackage{footnote}}
\makesavenoteenv{longtable}
\usepackage{graphicx}
\makeatletter
\newsavebox\pandoc@box
\newcommand*\pandocbounded[1]{% scales image to fit in text height/width
  \sbox\pandoc@box{#1}%
  \Gscale@div\@tempa{\textheight}{\dimexpr\ht\pandoc@box+\dp\pandoc@box\relax}%
  \Gscale@div\@tempb{\linewidth}{\wd\pandoc@box}%
  \ifdim\@tempb\p@<\@tempa\p@\let\@tempa\@tempb\fi% select the smaller of both
  \ifdim\@tempa\p@<\p@\scalebox{\@tempa}{\usebox\pandoc@box}%
  \else\usebox{\pandoc@box}%
  \fi%
}
% Set default figure placement to htbp
\def\fps@figure{htbp}
\makeatother


% definitions for citeproc citations
\NewDocumentCommand\citeproctext{}{}
\NewDocumentCommand\citeproc{mm}{%
  \begingroup\def\citeproctext{#2}\cite{#1}\endgroup}
\makeatletter
 % allow citations to break across lines
 \let\@cite@ofmt\@firstofone
 % avoid brackets around text for \cite:
 \def\@biblabel#1{}
 \def\@cite#1#2{{#1\if@tempswa , #2\fi}}
\makeatother
\newlength{\cslhangindent}
\setlength{\cslhangindent}{1.5em}
\newlength{\csllabelwidth}
\setlength{\csllabelwidth}{3em}
\newenvironment{CSLReferences}[2] % #1 hanging-indent, #2 entry-spacing
 {\begin{list}{}{%
  \setlength{\itemindent}{0pt}
  \setlength{\leftmargin}{0pt}
  \setlength{\parsep}{0pt}
  % turn on hanging indent if param 1 is 1
  \ifodd #1
   \setlength{\leftmargin}{\cslhangindent}
   \setlength{\itemindent}{-1\cslhangindent}
  \fi
  % set entry spacing
  \setlength{\itemsep}{#2\baselineskip}}}
 {\end{list}}
\usepackage{calc}
\newcommand{\CSLBlock}[1]{\hfill\break\parbox[t]{\linewidth}{\strut\ignorespaces#1\strut}}
\newcommand{\CSLLeftMargin}[1]{\parbox[t]{\csllabelwidth}{\strut#1\strut}}
\newcommand{\CSLRightInline}[1]{\parbox[t]{\linewidth - \csllabelwidth}{\strut#1\strut}}
\newcommand{\CSLIndent}[1]{\hspace{\cslhangindent}#1}



\setlength{\emergencystretch}{3em} % prevent overfull lines

\providecommand{\tightlist}{%
  \setlength{\itemsep}{0pt}\setlength{\parskip}{0pt}}



 


\usepackage{booktabs}
\usepackage{caption}
\usepackage{longtable}
\usepackage{colortbl}
\usepackage{array}
\usepackage{anyfontsize}
\usepackage{multirow}
\raggedright
\usepackage{tocloft}
\usepackage{float}
\usepackage{placeins}
\usepackage{needspace}
\setlength\LTpre{0pt}
\setlength\LTpost{0pt}
\KOMAoption{captions}{tableheading}
\makeatletter
\@ifpackageloaded{caption}{}{\usepackage{caption}}
\AtBeginDocument{%
\ifdefined\contentsname
  \renewcommand*\contentsname{Table of contents}
\else
  \newcommand\contentsname{Table of contents}
\fi
\ifdefined\listfigurename
  \renewcommand*\listfigurename{List of Figures}
\else
  \newcommand\listfigurename{List of Figures}
\fi
\ifdefined\listtablename
  \renewcommand*\listtablename{List of Tables}
\else
  \newcommand\listtablename{List of Tables}
\fi
\ifdefined\figurename
  \renewcommand*\figurename{Figure}
\else
  \newcommand\figurename{Figure}
\fi
\ifdefined\tablename
  \renewcommand*\tablename{Table}
\else
  \newcommand\tablename{Table}
\fi
}
\@ifpackageloaded{float}{}{\usepackage{float}}
\floatstyle{ruled}
\@ifundefined{c@chapter}{\newfloat{codelisting}{h}{lop}}{\newfloat{codelisting}{h}{lop}[chapter]}
\floatname{codelisting}{Listing}
\newcommand*\listoflistings{\listof{codelisting}{List of Listings}}
\makeatother
\makeatletter
\makeatother
\makeatletter
\@ifpackageloaded{caption}{}{\usepackage{caption}}
\@ifpackageloaded{subcaption}{}{\usepackage{subcaption}}
\makeatother
\usepackage{bookmark}
\IfFileExists{xurl.sty}{\usepackage{xurl}}{} % add URL line breaks if available
\urlstyle{same}
\hypersetup{
  pdftitle={Analysis of Potential Subgroups in Vaes ME/CFS Patient Clusters},
  pdfauthor={Erik Squires},
  colorlinks=true,
  linkcolor={blue},
  filecolor={Maroon},
  citecolor={Blue},
  urlcolor={Blue},
  pdfcreator={LaTeX via pandoc}}


\title{Analysis of Potential Subgroups in Vaes ME/CFS Patient Clusters}
\author{Erik Squires}
\date{2025-09-13}
\begin{document}
\maketitle


Affiliation: Independent Researcher

ORCID: 0009-0000-3843-4953

Email: esquires.research@proton.me

\section*{Abstract}\label{abstract}
\addcontentsline{toc}{section}{Abstract}

\textbf{Background:} Vaes et al.~(2023)\textsuperscript{1} identified 13
symptom clusters in a large cohort of ME/CFS patients. Symptom intensity
is broadly correlated with post-exertional malaise (PEM) severity, with
variation across clusters that seems disorganized. Despite this
research, no broadly accepted organizing principle has emerged from this
paper or other attempts at phenotyping ME/CFS.

\textbf{Objective:} To identify and characterize potential subgroups
defined by symptom domain severity relative to PEM within the original
Vaes symptom clusters.

\textbf{Methods:} We analyzed the Vaes cluster summary
data\textsuperscript{2}, calculated geometric means for each symptom
domain within each cluster, and plotted these means against PEM severity
to identify patterns and subgroups.

\textbf{Results:} Our analysis identified two groups of patient clusters
with distinct symptom-domain profiles. Anchoring symptom domains to PEM
collapses the 13 Vaes clusters into two reproducible families: one
characterized by parallel offsets (Autonomic, Neuroendocrine, Other) and
one by amplified slopes (Pain, Neurocognitive), with convergence at
higher PEM levels after accounting for a single influential cluster.
This subgroup's symptom trends could align with those of fibromyalgia.
Further exploration with individual patient data is needed to validate
these findings.

\textbf{Keywords:} Myalgic Encephalomyelitis, ME/CFS, patient
clustering, fibromyalgia, post-exertional malaise, PEM

\newpage{}

\setcounter{tocdepth}{2}
\tableofcontents

\newpage{}

\listoffigures
\listoftables

\newpage{}

\section{Introduction}\label{introduction}

Dr.~Anouk W. Vaes and her collaborators at CIRO performed one of the
largest systematic surveys of ME/CFS symptom patterns, applying
clustering methods to patient-reported outcomes to identify recurring
constellations of symptoms\textsuperscript{1}. They used symptom surveys
based on the DePaul Symptom Questionnaire version 2 (DSQ-2) from 337
patients to identify a total of 45 patient clusters, of which 13 were of
size \(\geq\) 10 while the remaining were of size \(\leq\) 2 and
approximately 9.4\% of the total cohort. We used the publicly available
data of their final 13 clusters for most of the
analysis\textsuperscript{2} but we discuss the small clusters in
Section~\ref{sec-small}.

While Vaes described each cluster, and noted some differences among
symptom severity, an overall organizing principle was not proposed. Here
we examine whether one might emerge from that dataset.

Our analysis suggests that the Vaes ME/CFS clusters can be organized
into two overarching families defined by their relationship to PEM. This
reduces the complexity of the cluster symptom intensities and highlights
a subgroup within the Vaes clusters that has not been explicitly
recognized in prior work. Once separated and profiled, this subgroup
shows amplified PEM per symptom domain, consistent with current research
on ME/CFS and fibromyalgia comorbidity. This paper invites further
investigation with patient-level symptom data to validate or refute
these findings and clarify their clinical and biological relevance.

Our central contribution is a PEM-referenced decomposition that reduces
the Vaes clusters to two linear families (offsets vs.~amplification) and
explains much of the apparent heterogeneity.

\section{Symptom Domains Relative to PEM}\label{sec-symptoms}

The strong weight of PEM symptoms makes the Vaes dataset tricky to
interpret if you only look at raw intensity, so instead we compared
overall symptom levels to PEM severity. From this comparison we
identified two groups of clusters that maintain their integrity across
most symptom domains. This PEM-anchored view is the organizing principle
for all results that follow.

Because these analyses are exploratory and based on cluster-level data,
the patterns described below should be interpreted as
hypothesis-generating.

We first examine the areas of most similarity before highlighting how
they differ. We add an ``all'' category to gauge overall symptom
severity to the original symptom domains. Our use of the terms
``high-intensity'' and ``low-intensity'' does not reproduce the clusters
described by Vaes or others but reflects a new organization based on how
symptom domains scale with PEM. Vaes used the DePaul symptom groupings
which we keep as-is.

\textbf{Notes:} The DSQ-2 uses a single symptom for Fatigue. Also, for
the ``Overall'' chart we used the arithmetic mean instead of geometric
mean for the Y axis. All other graphs in this study use geometric means
exclusively. Shaded bands represent the 95\% confidence interval of the
linear models when used. While this section focuses on visual
inspection, we provide more statistical descriptions in
\hyperref[sec-stats]{Appendix B}.

\subsection{Similarities}\label{sec-similar}

In terms of Fatigue, the two groups overlap completely, as ME/CFS
patient clusters are expected to. See Figure~\ref{fig-fatigue}, below.
While the lines have a similar slope and offset we note the R² value is
significantly weaker in the low-intensity group (0.54 vs.~0.86).

\begin{figure}[h]

\centering{

\includegraphics[width=1\linewidth,height=\textheight,keepaspectratio]{VaesSubgroups_files/figure-pdf/fatigue-1.pdf}

}

\caption{\label{fig-fatigue}Fatigue vs.~PEM}

\end{figure}%

\FloatBarrier

While Fatigue shows complete overlap in our groups the next charts
(Figure~\ref{fig-symptoms}) show how they stand out. In the first chart
(top left) we compare the average of all symptoms to PEM. In this chart
one can easily discern two tiers of clusters. One has a higher overall
symptom burden at any given PEM level compared to the other and this
relationship is where we derive the names for our groups: high and
low-intensity.

Autonomic, Neuroendocrine and Other also follow a similar pattern:
Parallel but lower than the high-intensity group.

We note that the limited range of PEM severity in the low-intensity
group (\(\approx 2.2\) to 2.9) may have otherwise caused it to remain
undetected.

\begin{figure}[h]

\centering{

\includegraphics[width=1\linewidth,height=\textheight,keepaspectratio]{VaesSubgroups_files/figure-pdf/all_symptoms-1.pdf}

}

\caption{\label{fig-symptoms}Similar Symptom Domains vs.~PEM}

\end{figure}%

\FloatBarrier

\textbf{Groupings:} From visual inspection of the charts above we group
clusters C9, C19, C26, C28, C31, C36, C37, C40 as high-intensity, C2,
C4, C7, C11, C24 as low-intensity. We'll keep these groupings for all
plots that follow. The high and low groups are approximately 54.5\% and
45.5\% of the total cohort, respectively. We further discuss our
grouping choices in \hyperref[sec-loo]{Leave-one-out Analysis}. We group
C36 with the high-intensity clusters, but it is also an outlier for the
entire cohort, and discuss this further in Section~\ref{sec-different}.

The last chart (Other vs.~PEM) shows a markedly lower offset and is
perhaps the most visibly obvious difference between the two groups.
Unfortunately, `Other' only has two symptoms: sensitivity to mold and
vibration which might make it a poor candidate for per-patient
validation.

\FloatBarrier

While the groups maintain integrity, the parallelism observed in the
charts above does not hold in the next charts and suggests a more
complex physiological cause for the differences.

\subsection{Converging Patterns Relative to PEM}\label{sec-different}

Pain and Neurocognitive domains show that while the low-intensity group
remains offset, as PEM severity increases to 3, the two lines converge.
Arguably, this is either increased amplification, or a reduction of
differences around C26. This distinct convergence between the two
cluster groups does not appear in any of the other symptom domains.

\begin{figure}[h]

\centering{

\includegraphics[width=1\linewidth,height=\textheight,keepaspectratio]{VaesSubgroups_files/figure-pdf/neuro_c_symptoms-1.pdf}

}

\caption{\label{fig-pain}Pain and Neurocognitive vs.~PEM}

\end{figure}%

\FloatBarrier

Without C36, the cohort outlier, the upper limits of PEM, Neurocognitive
and Pain would converge around where C7 (low-intensity) and C26
(high-intensity) meet. Vaes et al.~note that the participants in C36
``had the highest symptom burden (i.e.~highest frequency and severity of
symptoms)''\textsuperscript{1} and all of the charts in this study are
consistent with that observation.

\subsection{Weak Correlations Relative to
PEM}\label{weak-correlations-relative-to-pem}

In all of the previous charts we've shown that both groups maintain a
strong but distinct relationship to PEM. By contrast, the Immune and
Sleep domains show little to no correlation with PEM severity in the
low-intensity group. Notably, despite this lack of correlation, sleep
disturbance can be pronounced in the low-intensity group.

\begin{figure}[h]

\centering{

\includegraphics[width=1\linewidth,height=\textheight,keepaspectratio]{VaesSubgroups_files/figure-pdf/immune_symptoms-1.pdf}

}

\caption{\label{fig-distinct}Immune and Sleep vs.~PEM}

\end{figure}%

\FloatBarrier

\subsection{Summary of Differences}\label{summary-of-differences}

\textbf{High-intensity:} Has a consistent amplification pattern: as PEM
increases, all other symptom domains rise together.

\textbf{Low-intensity:} Presents a more selective profile. Pain, and
Neurocognitive symptoms start low but converge with the high-intensity
group, while Sleep and Immune symptoms show low correlation.
Additionally, the low-intensity group is notably nestled in the middle
of the overall PEM range which could allow it to hide among the other
clusters if not actively searched for.

We summarize these differences in the table below.\bigskip

\begin{longtable}[]{@{}
  >{\raggedright\arraybackslash}p{(\linewidth - 4\tabcolsep) * \real{0.2787}}
  >{\raggedright\arraybackslash}p{(\linewidth - 4\tabcolsep) * \real{0.2131}}
  >{\raggedright\arraybackslash}p{(\linewidth - 4\tabcolsep) * \real{0.5082}}@{}}
\caption{Summary of Group Differences}\label{tbl-t1}\tabularnewline
\toprule\noalign{}
\begin{minipage}[b]{\linewidth}\raggedright
Feature
\end{minipage} & \begin{minipage}[b]{\linewidth}\raggedright
High group
\end{minipage} & \begin{minipage}[b]{\linewidth}\raggedright
Low group
\end{minipage} \\
\midrule\noalign{}
\endfirsthead
\toprule\noalign{}
\begin{minipage}[b]{\linewidth}\raggedright
Feature
\end{minipage} & \begin{minipage}[b]{\linewidth}\raggedright
High group
\end{minipage} & \begin{minipage}[b]{\linewidth}\raggedright
Low group
\end{minipage} \\
\midrule\noalign{}
\endhead
\bottomrule\noalign{}
\endlastfoot
Overall vs.~PEM & Higher & Lower - esp.~``Other'' \\
Fatigue & R² - 0.86 & Weaker R² - 0.54 \\
PEM range & \(\approx 1.3\) -- 3.4 & Narrower \(\approx 2.2\) -- 2.9 \\
Neurocognitive & Tracks PEM & Elevated amplification \\
Pain & Tracks PEM & Elevated amplification \\
Immune & Tracks PEM & Flat at \(\approx 1\) \\
Sleep & Tracks PEM & \(\approx 1.4\) -- 2.5, but uncorrelated \\
\end{longtable}

\bigskip

\textbf{Symptom Severity:} In terms of overall symptom range the two
groups largely overlap. The high-intensity group, by symptom intensity
alone, could be considered a superset of the low-intensity group. It is
the differing relationships to PEM and the selective amplification of
Pain and Neurocognitive symptom domains in the low-intensity group that
set the two groups apart. We discuss this further in
Section~\ref{sec-li}, below.

\section{Exploring the Low-intensity Subgroup}\label{sec-li}

Based on the trends noted in Section~\ref{sec-symptoms}, a discussion of
whether the low-intensity group could relate to fibromyalgia is
warranted. We note that Vaes et al.\textsuperscript{1} did not model
fibromyalgia diagnoses, nor did they report excluding patients with
fibromyalgia from their cohort. We cannot assert that the subgroup is
clinically fibromyalgia, only that its domain profile aligns with what
would be expected if fibromyalgia were present within this cohort.

Patterns in the low-intensity group have higher PEM burdens for all
symptom domains except Immune and Sleep, which are uncorrelated with
PEM. Further, Neurocognitive and Pain regression lines are amplified.

Vaes et al.\textsuperscript{1} recruited participants who self-reported
ME/CFS and later mapped them to multiple case definitions, yielding a
broad spectrum of clinical presentations:

\begin{quote}
Almost 90\% of the participants fulfilled the Fukuda case definition,
compared to 80\%, 59\% and 39\% fulfilling the IOM, CCC and ME-ICC case
definitions, respectively. More than a quarter of the participants met
the criteria for all four different case definitions, whilst 5\% of the
participants met none of the abovementioned case definitions,\ldots{}
\end{quote}

This breadth makes mixed phenotypes plausible. Independent evidence also
indicates that fibromyalgia frequently co-occurs with
ME/CFS\textsuperscript{3,4}, and comorbidity has been associated with
more frequent and severe PEM in ME/CFS patients\textsuperscript{5},
which could explain the subgroup's right-shift along the PEM axis.

Because the Vaes analysis used cluster-level summaries and did not model
comorbidities, we cannot identify fibromyalgia directly here.
Confirmation would require patient-level data we do not have access to.

\section{Characteristics of the Smallest Clusters}\label{sec-small}

Vaes et al.\textsuperscript{1} identified 13 large\textsuperscript{2}
and 32 small clusters\textsuperscript{6}. The small clusters were mostly
of size 1, with 4 clusters of size 2 and represent approximately 9.4\%
of the cohort. We examined both sets in light of our two-group
hypothesis and with similar charts to those in
Section~\ref{sec-symptoms}. Our goal was to examine if the small
clusters could illuminate or challenge our two-group hypothesis.

Overlaying both datasets, for most domains the small clusters (of size
n=1 or 2) spread above and below our two subgroup trend lines. This may
suggest that the two groups in our visualizations are not distinct but
artifacts of the clustering constraints. If this reflects a continuous
distribution instead of two, our PEM-referenced framework would still
provide a useful characterization of how symptoms vary across the
severity spectrum.

The Sleep domain stood out in comparison to the rest. The small clusters
mostly had Sleep intensity scores above the high-intensity group's
regression lines.

Finally, the small clusters, when added to the large clusters, did not
significantly move their location or regression lines.

\section{Discussion}\label{sec-discuss}

We have shown that the Vaes clusters can be organized into two subgroups
defined by symptom domains relative to PEM, and that these subgroups
remain consistent across most domains. Grouping them by their
relationship to PEM simplifies the original complexity at both the
domain and cluster levels. From this perspective, the 13 Vaes clusters
can be reduced to two groups of linear models, where membership in one
group combined with PEM intensity allows inference of the intensity of
other symptom domains. We view this PEM-referenced, two-family reduction
(offsets vs.~amplification, with C36 sensitivity at the high end) as the
main conceptual advance over prior descriptions of the Vaes clusters.

While these patterns are striking, our analysis is a secondary
exploration based on symptom-domain summaries of the Vaes patient data.
The small subgroup sizes (eight and five clusters) and the original
clustering approach may distort some relationships, and the plots can
suggest trends that do not necessarily exist at the patient level. The
apparent high/low separation may therefore reflect artifacts of the
clustering algorithm averaging a broad underlying distribution rather
than true biological subgroups, which would render our suggested
distinctions essentially coincidental. The framework is compatible with
fibromyalgia enrichment as a downstream hypothesis, but the
PEM-referenced structure stands independently of comorbidity.

These considerations highlight the need for validation before our
findings can be clinically relevant. We see challenges in at least three
areas: whether the groupings hold when tested with individual patient
data; whether symptom-domain analysis can reliably identify the group to
which a given patient belongs; and how these subgroups relate---or fail
to relate---to fibromyalgia.

As an example, Vaes\textsuperscript{1} includes cluster-specific
summaries of standout symptoms; some of those align with our
domain-based charts while others show a different emphasis,
demonstrating the challenge of reconciling symptom-level observations
with domain-level analyses.

Despite these limitations, we believe further steps are justified.
Analyses using the original patient scores could validate, refine, or
repudiate these subgroups and help identify a short list of
differentiating symptoms. If that proves useful, examining biological
markers across these subgroups could help reveal underlying
pathophysiological differences. In addition, applying similar analyses
to fibromyalgia and long-COVID could clarify relationships among these
overlapping conditions or even help refine how ME/CFS and fibromyalgia
are classified.

\section{Conclusion}\label{conclusion}

Our secondary analysis suggests that the 13 clusters identified by Vaes
et al.~can be organized into two broader structures defined by overall
symptom intensity relative to PEM severity. These two groups remain
coherent across symptom domains and display distinct, largely linear
relationships with PEM. Although exploratory and limited to
cluster-level data, these findings offer a reproducible framework for
further research, specifically for validating these potential subgroups
and exploring their clinical and biological relevance.

\clearpage

\section{Appendices}\label{appendices}

\FloatBarrier

\subsection*{Appendix A: Cluster Summary Data}\label{sec-cluster-data}
\addcontentsline{toc}{subsection}{Appendix A: Cluster Summary Data}

We use the Vaes cluster spreadsheet\textsuperscript{2} exclusively as
our data source. We use R for significant cleaning and transformation
from Excel to CSV. After removing everything but the mean severity of
the symptoms in the 13 clusters we:

\begin{itemize}
\tightlist
\item
  Group each cluster by symptom domain and calculate geometric means for
  each symptom domain within each cluster.
\item
  Rotate the table
\item
  Add a new column, ``all\_mean'' which is the arithmetic mean of all
  symptoms within each cluster.
\item
  Save this file as cluster\_grouped\_tidy.csv
\end{itemize}

This file is what we then used for our analysis, and the plots. We
present the final data used below. Values were rounded and abbreviations
were used for formatting. See the \hyperref[sec-code]{Code Availability}
section for links to the code used to generate this file.

\begin{table}[h]

\caption{\label{tbl-t3}Cluster Summary Data with Geometric Means by
Symptom Domain}

\centering{

\fontsize{12.0pt}{14.4pt}\selectfont
\begin{tabular*}{\linewidth}{@{\extracolsep{\fill}}lrrrrrrrrr}
\toprule
Cluster & All & PEM & Fatigue & Auto & Immune & NC & NE & Pain & Other \\ 
\midrule\addlinespace[2.5pt]
C2 & 1.21 & 2.30 & 2.79 & 0.76 & 0.53 & 1.15 & 0.67 & 0.95 & 0.27 \\ 
C4 & 1.55 & 2.40 & 3.13 & 1.24 & 0.87 & 1.66 & 0.96 & 1.19 & 0.46 \\ 
C7 & 1.92 & 2.85 & 3.20 & 1.48 & 1.03 & 2.32 & 1.05 & 1.94 & 0.86 \\ 
C9 & 2.06 & 2.64 & 2.85 & 2.16 & 1.40 & 2.24 & 1.52 & 1.71 & 1.55 \\ 
C11 & 1.42 & 2.23 & 2.94 & 1.02 & 1.08 & 1.11 & 0.64 & 1.00 & 0.33 \\ 
C19 & 1.37 & 1.97 & 2.53 & 1.12 & 0.57 & 1.81 & 0.61 & 1.06 & 0.60 \\ 
C24 & 1.63 & 2.50 & 2.95 & 1.44 & 0.82 & 1.52 & 0.86 & 1.53 & 0.56 \\ 
C26 & 2.14 & 2.80 & 3.28 & 1.87 & 1.76 & 2.06 & 1.52 & 2.34 & 1.77 \\ 
C28 & 1.03 & 1.62 & 2.50 & 0.83 & 0.56 & 0.78 & 0.60 & 0.88 & 0.30 \\ 
C31 & 1.47 & 2.15 & 2.73 & 1.31 & 0.77 & 1.31 & 1.04 & 1.54 & 1.41 \\ 
C36 & 2.74 & 3.35 & 3.50 & 2.59 & 2.30 & 3.08 & 2.04 & 2.73 & 2.04 \\ 
C37 & 0.67 & 1.30 & 2.39 & 0.41 & 0.16 & 0.43 & 0.39 & 0.51 & 0.19 \\ 
C40 & 1.21 & 1.76 & 2.80 & 0.74 & 0.52 & 0.95 & 0.82 & 1.19 & 0.67 \\ 
\bottomrule
\end{tabular*}

}

\end{table}%

\FloatBarrier
\clearpage

\subsection*{Appendix B: Statistical Descriptions}\label{sec-stats}
\addcontentsline{toc}{subsection}{Appendix B: Statistical Descriptions}

In the body of the text, we used visual analysis to identify two
subgroups of ME/CFS patient clusters defined by their symptom domain
profiles relative to PEM severity. Here we present statistics that may
help clarify the data in the plots. Given the exploratory nature of this
study and the small number of clusters, these results should be
interpreted as descriptive rather than confirmatory.

\bigskip

\subsubsection*{ANCOVA}\label{sec-ancova}
\addcontentsline{toc}{subsubsection}{ANCOVA}

We performed an analysis of covariance (ANCOVA) to identify the domains
which could be most statistically significant and discriminatory. We
compare our two groups for each symptom domain. We arrange the results
by descending F-statistic and show the p-value rounded to three decimal
places.

\FloatBarrier

\begin{table}[h]

\caption{\label{tbl-t6}ANCOVA Analysis}

\centering{

\fontsize{12.0pt}{14.4pt}\selectfont
\begin{tabular*}{\linewidth}{@{\extracolsep{\fill}}lrlr}
\toprule
Domain & F statistic & p-value & Significance \\ 
\midrule\addlinespace[2.5pt]
Other & 33.3 & < 0.001 & *** \\ 
Neuroendocrine & 11.6 & 0.007 & ** \\ 
Fatigue & 5.2 & 0.046 & * \\ 
Autonomic & 3.8 & 0.08 & . \\ 
Pain & 3.1 & 0.108 & NA \\ 
Immune & 1.4 & 0.268 & NA \\ 
All\_mean & 1.0 & 0.33 & NA \\ 
Sleep & 0.2 & 0.647 & NA \\ 
Neurocognitive & 0.0 & 0.849 & NA \\ 
\bottomrule
\end{tabular*}

}

\end{table}%

Table~\ref{tbl-t6} shows agreement with the charts that Other,
Neuroendocrine and Autonomic domains show the most significant offsets
between the two groups. The high F-statistic and low p-value for
Fatigue, however, do not agree with the regression lines. As we note
earlier in Section~\ref{sec-similar}, the lines in
Figure~\ref{fig-fatigue} are nearly identical but R² is significantly
weaker in the low-intensity group. The high F-statistic and low p-value
in the ANCOVA analysis for fatigue is most likely an artifact of the
wide spread and small sample count in the low-intensity group.

\FloatBarrier

\subsubsection*{Leave-one-out Analysis}\label{sec-loo}
\addcontentsline{toc}{subsubsection}{Leave-one-out Analysis}

We discovered our two groups by visual analysis, but we can re-identify
the set using leave-one-out analysis. In this analysis we go through
each domain and examine what happens when we remove one cluster at a
time. The goal is to identify which clusters, when removed, improve R²
the most.

We summarize the mean R² differences, in order of greatest improvement,
in Table~\ref{tbl-t5}.

\begin{table}[h]

\caption{\label{tbl-t5}Leave-one-out Analysis}

\centering{

\fontsize{12.0pt}{14.4pt}\selectfont
\begin{tabular*}{\linewidth}{@{\extracolsep{\fill}}lcrr}
\toprule
Cluster & Group & Domains Improved & Mean ΔR² \\ 
\midrule\addlinespace[2.5pt]
C2 & low & 9 & 0.0379 \\ 
C11 & low & 9 & 0.0271 \\ 
C4 & low & 8 & 0.0191 \\ 
C31 & high & 7 & 0.0163 \\ 
C7 & low & 5 & 0.0129 \\ 
C24 & low & 8 & 0.0126 \\ 
\bottomrule
\end{tabular*}

}

\end{table}%

Of the top 6 clusters only C31 is not part of the low-intensity group.
While this does not prove the existence of two subgroups it does support
the idea that the members of the low-intensity group are, overall,
responsible for most of the R² degradation in the combined group.

C31 does exhibit variability but it varies around the high-intensity
group trend lines, which is why we include it with the high group
despite this result.

\FloatBarrier 
\bigskip

\subsubsection*{R² by Group}\label{sec-r2-group}
\addcontentsline{toc}{subsubsection}{R² by Group}

We compare the coefficient of determination (R²) of the symptom domains
in three levels:

\begin{itemize}
\tightlist
\item
  Combined (all clusters)
\item
  High-intensity group
\item
  Low-intensity group
\end{itemize}

\FloatBarrier

\begin{table}[h]

\caption{\label{tbl-t4}R² by Domain}

\centering{

\fontsize{12.0pt}{14.4pt}\selectfont
\begin{tabular*}{\linewidth}{@{\extracolsep{\fill}}lrrr}
\toprule
Domain & Combined R² & High-intensity R² & Low-intensity R² \\ 
\midrule\addlinespace[2.5pt]
All\_mean & 0.91 & 1.00 & 0.84 \\ 
Autonomic & 0.79 & 0.96 & 0.62 \\ 
Fatigue & 0.84 & 0.86 & 0.54 \\ 
Immune & 0.78 & 0.96 & 0.11 \\ 
Neurocognitive & 0.85 & 0.92 & 0.93 \\ 
Neuroendocrine & 0.73 & 0.95 & 0.75 \\ 
Other & 0.52 & 0.92 & 0.96 \\ 
Pain & 0.81 & 0.95 & 0.95 \\ 
Sleep & 0.53 & 0.82 & 0.05 \\ 
\bottomrule
\end{tabular*}

}

\end{table}%

Table~\ref{tbl-t4} shows that for almost every symptom domain, the
high-intensity group has a higher R² than the combined group, which in
turn has a higher R² than the low-intensity group. In some cases -- such
as Neurocognitive, Neuroendocrine, and Other -- the combined R² is
significantly weaker than either of the subgroups.

\FloatBarrier
\bigskip

\section*{Code Availability}\label{sec-code}
\addcontentsline{toc}{section}{Code Availability}

All data manipulation and analysis scripts are available at
\url{https://github.com/eriksquires/VaesSubgroups}

\section*{Data Availability}\label{sec-data}
\addcontentsline{toc}{section}{Data Availability}

A copy of the original Vaes spreadsheets are available in the GitHub
repository, above, and from the original Vaes
publication\textsuperscript{1} at
\url{https://doi.org/10.1186/s12967-023-03946-6}.

All other data files are created from the original Vaes spreadsheets
using the code in the GitHub repository, above.

\newpage{}

\section*{Acknowledgements}\label{acknowledgements}
\addcontentsline{toc}{section}{Acknowledgements}

This paper could not exist without the foundational work of Dr.~Anouk W.
Vaes and her colleagues at CIRO, whose clustering
study\textsuperscript{1} and publicly available cluster summary
data\textsuperscript{2} provided the basis for our analysis. We are
deeply grateful for their contribution to the field.

Dr.~Vaes had no role in the writing of this paper, and all errors or
misinterpretations are the responsibility of the author.

\section*{Author Contributions}\label{author-contributions}
\addcontentsline{toc}{section}{Author Contributions}

Erik K. Squires conceived the study, performed the analysis, and wrote
the manuscript. This work presents an original investigative method and
resulting framework which were both developed and first reported by the
author in this preprint.

\section*{Competing Interests}\label{competing-interests}
\addcontentsline{toc}{section}{Competing Interests}

The author declares no competing interests.

\section*{Funding}\label{funding}
\addcontentsline{toc}{section}{Funding}

This work received no external funding.

\section*{Ethics Statement}\label{ethics-statement}
\addcontentsline{toc}{section}{Ethics Statement}

This study reanalyzed publicly available reports and published symptom
cluster data (Vaes 2023). No new patient data were collected.

\newpage{}

\section*{References}\label{references}
\addcontentsline{toc}{section}{References}

\phantomsection\label{refs}
\begin{CSLReferences}{0}{0}
\bibitem[\citeproctext]{ref-vaes2023}
\CSLLeftMargin{1. }%
\CSLRightInline{Vaes, A. W. \emph{et al.}
\href{https://doi.org/10.1186/s12967-023-03946-6}{Symptom-based clusters
in people with ME/CFS: An illustration of clinical variety in a
cross-sectional cohort}. \emph{Journal of Translational Medicine}
\textbf{21}, 112 (2023).}

\bibitem[\citeproctext]{ref-vaes2023_excel_large}
\CSLLeftMargin{2. }%
\CSLRightInline{Vaes, A. W. \& collaborators.
\href{https://static-content.springer.com/esm/art\%3A10.1186\%2Fs12967-023-03946-6/MediaObjects/12967_2023_3946_MOESM3_ESM.xlsx}{Supplementary
dataset (excel): Additional file 3: Table S1 :
12967\_2023\_3946\_MOESM3\_ESM.xlsx}. Journal of Translational Medicine,
Springer Nature (2023).}

\bibitem[\citeproctext]{ref-fall2024mcam}
\CSLLeftMargin{3. }%
\CSLRightInline{Fall, E. A. \emph{et al.}
\href{https://doi.org/10.1186/s12883-024-03872-0}{Chronic overlapping
pain conditions in people with myalgic encephalomyelitis/chronic fatigue
syndrome (ME/CFS): A sample from the multi-site clinical assessment of
ME/CFS (MCAM) study}. \emph{BMC Neurology} \textbf{24}, (2024).}

\bibitem[\citeproctext]{ref-ramirez2022overlap}
\CSLLeftMargin{4. }%
\CSLRightInline{Ramı́rez-Morales, R., Bermúdez-Benı́tez, E.,
Martı́nez-Martı́nez, L.-A. \& Martı́nez-Lavı́n, M.
\href{https://doi.org/10.1016/j.autrev.2022.103129}{Clinical overlap
between fibromyalgia and myalgic encephalomyelitis: A systematic review
and meta-analysis}. \emph{Autoimmunity Reviews} \textbf{21}, 103129
(2022).}

\bibitem[\citeproctext]{ref-mcmanimen2017pem}
\CSLLeftMargin{5. }%
\CSLRightInline{McManimen, S. L. \& Jason, L. A.
\href{https://www.ncbi.nlm.nih.gov/pmc/articles/PMC5464757}{Post-exertional
malaise in patients with ME and CFS with comorbid fibromyalgia}.
\emph{SRL Neurology \& Neurosurgery} \textbf{3}, 22--27 (2017).}

\bibitem[\citeproctext]{ref-vaes2023_excel_small}
\CSLLeftMargin{6. }%
\CSLRightInline{Vaes, A. W. \& collaborators.
\href{https://static-content.springer.com/esm/art\%3A10.1186\%2Fs12967-023-03946-6/MediaObjects/12967_2023_3946_MOESM4_ESM.xlsx}{Supplementary
dataset (excel): Additional file 4 table S2 :
12967\_2023\_3946\_MOESM4\_ESM.xlsx}. Journal of Translational Medicine,
Springer Nature (2023).}

\end{CSLReferences}




\end{document}
